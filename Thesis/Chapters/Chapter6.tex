\chapter{Conclusion}
Although SS 433 have been studied for almost 40 years, it still remains a mysterious astrophysical source.
We observed SS 433 with Chandra, an advance X-ray telescope to explore this enigmatic source. Using the High Energy Transmission Grating System, we took a total of 116 ksec observation, which contains a 20 ksec short observation and a 96 ksec long observation at a new combination of orbital and jet precession phases. This special strategy enables us to map out the spatial variation of the jet's properties, such as temperature and ionization states along the jet. Through fitting two phenomenological models to the spectrum, we were able to probe the changes of the properties of emission lines from both jets between the two observations and within the long observation. The plasma model enables us the probe the physical conditions in the jet base.\par 

The unexpected redshift observed in the Western jet reinforce the hypothesis of the previous paper that the Eastern and Western jets were moving independent to each other. The varying normalization may suggest a changing intrinsic accretion flow during the observation. The unexpected line flux of the Western jet during the eclipse and the photon index of the power law also give some insights to the size of the donor star, which could be further studied in the future. \par 
This work gives rise to numerous questions that could be addressed in future studies of SS 433, both observational as well as theoretical. Plasma models for different parts in the long observation could be fit and compared so that we can probe physical condition changes of the jet during the eclipse more carefully. Better plasma models could also be developed to fit the natural properties of the jet more accurately. Therefore, this is not the end of this research project. We plan to continue researching on this observation of SS 433 to refine our approach to analyze this subject and get a better understanding to it.