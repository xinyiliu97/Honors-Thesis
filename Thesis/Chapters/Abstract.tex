\chapter{Abstract}
The Galactic X-ray binary SS 433 is the only known astrophysical object to exhibit strong, relativistically red- and blue-shifted lines from elements such as S, Si, Fe, Ni. The X-ray emission lines originate in a jet outflow that is launched somewhere very close to the compact accretor (a black hole or a neutron star). During 2018 August 10-14, SS 433 was observed using the High Energy Transmission Grating Spectrometer system on the Chandra X-ray Observatory. A total of 116 ksec of Chandra HETGS observation was made during 2018 August 10-14, which was split in one 20 ksec and one 96 ksec observation. The 20 ksec observation started at an orbital phase of 0.802, three days before an eclipse and the 96 ksec observation started at an orbital phase of 0.109 in the middle of the eclipse. The observations were designed to take advantage of the eclipse and carry out time-resolved spectroscopy and timing studies to infer spatial variation of physical properties such as composition, temperature, and density at different distances along the jet. In addition to phenomenological fits to determine properties of the observed emission lines, we will present results from fitting collisionally ionized plasma models. The observation indicates that the redshift of the Western jet was 0.034 while the predicted redshift value is -0.0097. This suggests the independent motion of the two jets. The disappearance of the Fe {\sc xxv} line from the Eastern jet and the high photon index of the power law at the end of the long observation suggest the accretor might not has come out of the eclipse. 